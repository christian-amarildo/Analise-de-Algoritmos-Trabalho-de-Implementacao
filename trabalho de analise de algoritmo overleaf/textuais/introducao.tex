\chapter{INTRODUÇÃO}
\label{chap:introducao }

A análise de algoritmos é um campo fundamental da ciência da computação, cujo objetivo é avaliar a eficiência e o desempenho de algoritmos em termos de tempo de execução e uso de recursos. Com o crescente volume de dados processados diariamente, a escolha do algoritmo adequado para uma determinada tarefa torna-se crucial para garantir a eficiência computacional. Nesse contexto, este trabalho tem como foco a análise comparativa de três algoritmos clássicos de ordenação: Quicksort, Heapsort e Shellsort, visando identificar suas vantagens, desvantagens e cenários de aplicação.

A ordenação de dados é uma operação básica em computação, utilizada em uma ampla gama de aplicações, desde sistemas de bancos de dados até processamento de imagens. Cada algoritmo de ordenação possui características próprias que o tornam mais adequado para determinados tipos de dados ou tamanhos de conjuntos. Portanto, compreender o comportamento desses algoritmos em diferentes cenários é essencial para a tomada de decisões informadas no desenvolvimento de software.

Este trabalho propõe a implementação dos algoritmos Quicksort, Heapsort e Shellsort em uma linguagem de programação, seguida de testes comparativos com diferentes massas de dados. Além disso, serão analisados aspectos como o número de comparações, trocas e o tempo de execução, com o intuito de fornecer uma visão clara sobre o desempenho de cada algoritmo. A partir dos resultados obtidos, serão elaborados gráficos e tabelas que ilustram as diferenças entre os métodos de ordenação, contribuindo para um entendimento mais profundo de suas complexidades.

\section{Objetivos}
\label{sec:objetivos }

\subsection{Objetivo geral}
\label{subsec:objetivogeral }

Realizar uma análise comparativa dos algoritmos de ordenação Quicksort, Heapsort e Shellsort, avaliando seu desempenho em termos de tempo de execução, número de comparações e trocas, com base em diferentes cenários de teste.

\subsection{Objetivos específicos}
\label{subsec:objetivosespecificos }

O projeto consiste em implementar os algoritmos de ordenação Quicksort, Heapsort e Shellsort em uma linguagem de programação, seguido da elaboração de massas de teste variadas para avaliar o desempenho desses algoritmos em diferentes cenários. Será realizada uma análise da complexidade temporal de cada algoritmo, considerando casos médios, melhores e piores, a fim de compreender seu comportamento em situações distintas. Os resultados obtidos serão comparados por meio de tabelas e gráficos, destacando as vantagens e desvantagens de cada método de ordenação. Além disso, o projeto inclui a resolução de problemas computacionais propostos, aplicando os conceitos estudados e calculando a complexidade dos algoritmos utilizados, proporcionando uma visão prática e teórica sobre a eficiência e aplicabilidade desses algoritmos em contextos reais.



\section{Estrutura do trabalho}
\label{sec:estrututaTrabalho }

Este trabalho está organizado em cinco seções, além das referências bibliográficas e anexos.

Na Seção 1, é apresentada a introdução, com o contexto do trabalho, os objetivos e a metodologia utilizada; a Seção 2 descreve a implementação dos algoritmos Quicksort, Heapsort e Shellsort, com os códigos-fonte e uma breve explicação de seu funcionamento; a Seção 3 apresenta os testes realizados, com a descrição dos arquivos de teste utilizados e a análise dos resultados obtidos; a Seção 4 aborda a resolução dos problemas computacionais propostos, com o cálculo da complexidade dos algoritmos aplicados; e, por fim, a Seção 5 traz as conclusões do trabalho, destacando os principais achados e sugerindo possíveis direções para pesquisas futuras.







































































































































% % INTRODUÇÃO-------------------------------------------------------------------

% \chapter{INTRODUÇÃO}
% \label{chap:introducao}

% Alguns programas podem ser utilizados para auxílio da escrita do TCC entre eles o \textit{MathType} (com relação a equações), \textit{Inkscape} (com relação a imagens). \\ %enter

% \textbf{PRIMEIRAS ORIENTAÇÕES}\\ %\textbf - negrito

% 1) O comando ``$\backslash$autoref\{label\}'' auto referencia o respectivo ``\textit{label}".

% Exemplo 1: De acordo com o exposto no \autoref{chap:introducao}... Pode-se verificar na \autoref{fig:consumomundialporfonte-fig1}...\\

% 2) O comando ``$\backslash$citeonline\{bibid\}'' é utilizado para citações diretas. Ele cita o respectivo ``\textit{bibid}".\\

% Exemplo 2:  Conforme \citeonline{AMARANTEMESQUITA20141261} cita em seu artigo, turbinas hidrocinéticas atualmente têm... \citeonline{VAZ2018509} também ressalta que turbinas de eixo horizontal possuem maiores...\\

% \citeonline{vallverdu2014}  --->  Vallverdú (2014) --- DIRETA

% \cite{vallverdu2014}   --->  (VALLVERDÚ, 2014) --- INDIRETA\\

% O comando ``$\backslash$cite\{bibid\}'' é utilizado para citações indiretas. Ele cita o respectivo ``\textit{bibid}".\\

% Exemplo 3: A máxima eficiência que uma turbina hidrocinética ideal pode alcançar é dada pelo Limite de Betz-Joukowski que corresponde a 59,3\%, o equivalente a um $C_P$ de 0,593 \cite{vallverdu2014, SHINOMIYA2015d}.\\ %Ao colocar a virgula, pode-se adicionar mais de um autor à citação.

% 3) Um ponto final é representado por um espaço entre os parágrafos.

% Exemplo 1.
% Exemplo 2.

% Exemplo 3.

% Exemplo 4.\\

% 4) Figuras

% Figuras com extensão .jpg, .pdf, .eps, .ps, .png

% As figuras devem ser adicionadas a pasta ``$\backslash$figuras'' no diretório deste template.

% %[!htb] - são as opções onde o LaTeX escolhe a melhor posição para inserir a figura na página, aqui (here), topo (top) ou embaixo (bottom), respectivamente. Se você colocar apenas um deles, por exemplo [!h], a figura ficará exatamente onde você inseriu.

% %Modelo de inclusão de figura. É só copiar, tomar como modelo e modificar.
% \begin{figure}[H]
% 	\centering
% 	\caption{Consumo mundial de energia por fonte de energia em quatrilhões de BTU.}
% 	\includegraphics[width=1.0\textwidth]{figuras/consumodeenergiamundialporfonte.pdf}
% 	\fonte{\citeonline{harris2006essential}.}
% 	\label{fig:consumomundialporfonte-fig1}
% \end{figure}

% 5) Referências

% As referências devem ser adicionadas no arquivo ``base-referencias.bib'' no diretório deste template. 

% Modelos podem ser editados na página ``https://truben.no/latex/bibtex/''. A partir do DOI pode-se encontrar o arquivo \textit{.bib} em ``https://www.doi2bib.org/''. No Google Acadêmico também se encontram bastantes referências no formato \textit{.bib}.

% Tome cuidado com autores com nomes que termiam em Júnior, Filho, Neto e etc. Forma correta: ``Fulano Deltrano Siclano\{ \}Neto''.

% As referências não reconhecem legal os pacotes de acentos. Então deve-se utilizar comandos de acentos. ``http://latexbr.blogspot.com/2011/02/acentos-e-caracteres-especiais.html''.

% *Ao ser executado pela primeira vez, possa ser que você precise está conectado a internet para o programa instalar os \textit{packages} necessários para compilar o arquivo PDF.

% Utilize esse \textit{template} sempre verificando as normas da Biblioteca Central da UFPA segundo o Guia para Elaboração de Trabalhos Acadêmicos disponível em http://bc.ufpa.br/ além das normas da ABNT.

% Outras orientações podem ser encontradas na internet.\\

% Boa escrita!

% \section{Objetivos}
% \label{sec:objetivos}

% \subsection{Objetivo geral}
% \label{subsec:objetivogeral}

% Escreva seu objetivo geral aqui.

% \subsection{Objetivos específicos}
% \label{subsec:objetivosespecificos}

% \begin{itemize}
% \item Escreva seu objetivo específico 1 aqui;
% \item Escreva seu objetivo específico 2 aqui;
% \item ...;
% \end{itemize}

% \section{Estrutura do trabalho}
% \label{sec:estrututaTrabalho}

% Este trabalho está dividido em cinco seções, referências, anexos e apêndices.

% Na seção 1 é apresentado o contexto no qual o trabalho está inserido, a justificativa e os objetivos almejados...

% A revisão bibliográfica sobre as temáticas relacionadas com essa pesquisa é apresentada na seção 2...

% A seção 3 mostra conceitos teóricos relacionados às ferramentas utilizadas no estudo tal como...

% Na seção 4, os resultados são apresentados juntamente com suas devidas discussões, verificando...

% Finalizando, a seção 5 faz as devidas conclusões e apresenta sugestões para trabalhos futuros.
